Therefore, using its procedure call sugar, $                                                                                                                                                                                                                                                                                  $,
we can prove the following refinement:
\begin{align*}
  & \PROC \ ISPRIME(\VALUE \ n, \RESULT \ r) \cdot \\
  & n, r: [TRUE, (n \in Prime \And r > 0) \Or (n \notin Prime \And r<=0)]\\
  \refstep{}
  {\IF~n \in Prime\\
  &\THEN~\ r:[r = ISPRIME(\VALUE \n)\And r>0, n\in Prime]\\
  &\ELSE~\nt{p:[l\neq r+1\And\pre(2),\post(2)]}{(4)}\\
  &\FI}\\
\end{align*}
Then we can base on the GMP function that make up the spec of the 
procedure call of $\PROC \ ISPRIME(\VALUE \ n, \RESULT \ r)$ 
\begin{gather*}
  \PROC \ ISPRIME(\VALUE \ n, \RESULT \ r) \cdot \\
  [TRUE, (n \in Prime \And r > 0) \Or (n \notin Prime \And r=0)]
\end{gather*}
Which means $ ISPRIME(n)>0 
\footnote{This act as another procedure call sugar $r \Ass ISPRIME(n)$;
the complete expanded version will be 
$ISPRIME(n,r);(( r>0 \Implies r \in Prime)
\Or (\ELSE \Implies r \notin Prime))$. The continueous prove 
will follow this convension to make life easier.
} 
\Implies n \in Prime $.

